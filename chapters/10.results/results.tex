\section{Tests on the MeerKAT LMC observation}\label{results}
For algorithmic testing


	
\begin{figure}[h]
	\centering
	\includegraphics[width=0.80\linewidth]{./chapters/10.results/LMC/meerkat.png}
	\caption{Radio interferometer system}
	\label{results:radio}
\end{figure}



\subsection{Comparison with CLEAN reconstruction}

\subsection{GPU Acceleration}

\subsection{Distributed coordinate descent}




We are in the realm of convolution. Remember that a convolution in image space is a multiplication in fourier space.
We can multiply

\subsection{Wall clock time}
\begin{figure}[h]
	\centering
	\includegraphics[width=0.80\linewidth]{./chapters/10.results/wall-clock-time.png}
	\caption{Wall-clock time of the distributed reconstruction}
	\label{results:time:fig}
\end{figure}


\subsection{Validity of gradient approximation} \label{results:gradients}

\begin{figure}[h]
	\centering
	\begin{subfigure}[b]{1.0\linewidth}
		\includegraphics[width=\linewidth]{./chapters/10.results/gradient/size.png}
	\end{subfigure}
	
	\caption{Effect of the L1 and L2 Norm separately.}
	\label{results:gradients:size}
\end{figure}

\begin{figure}[h]
	\centering
	\begin{subfigure}[b]{1.0\linewidth}
		\includegraphics[width=\linewidth]{./chapters/10.results/gradient/comparison.png}
	\end{subfigure}
	\begin{subfigure}[b]{1.0\linewidth}
		\includegraphics[width=\linewidth]{./chapters/10.results/gradient/comparison_zoom.png}
	\end{subfigure}
	
	\caption{Effect of the L1 and L2 Norm separately.}
	\label{results:gradients:comparison}
\end{figure}