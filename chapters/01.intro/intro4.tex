\section{The image reconstruction problem of radio interferometers}
In Radio Astronomy, the angular resolution of a single antenna is limited by the dish-diameter. Of course, there is a practical limit on the size of antenna-dishes we can build. Radio interferometers go around this problem. They use several smaller antennas together, acting like a single large dish. Several smaller dishes together can achieve higher angular resolution than single-dish instruments.

But there are drawbacks. First, the interferometer does not measure the sky in pixels. It measures the amplitude and phase of Fourier components at a given $u$ and $v$ location. The observed image has to be reconstructed from the measurements. And second, radio interferometers produce an ever increasing volume of Fourier measurements. Recently, the new MeerKAT radio interferometer was completed. It can produce roughly 80 million Fourier components every second, and a single observation may be conducted over several hours. Reconstructing an image for radio interferometers is therefore a big data problem.

At first glance, we might believe that the image reconstruction is trivial: The interferometer measures Fourier components, and efficient algorithms for the inverse Fourier transforms are known. However, two properties of the measured Fourier components make the image reconstruction difficult: The measurements are both noisy and incomplete.

The amplitude and phase of each measured Fourier component contains noise.
Radio waves on their way to earth experience distortions from various sources. and from the equipment. Changing conditions can create heavy interference, and the Fourier measurements are heavily corrupted by noise. Image reconstruction has to de-noise the measured Fourier components.

The measured Fourier components are incomplete. Although radio interferometers produce a large volume of measured Fourier components, they cannot measure every relevant Fourier component. There are Fourier components with non-zero amplitudes that the interferometer has not seen. The reconstruction algorithm has to find the observed image even though important Fourier components are missing from the measurements.

A reconstruction algorithm that can:
\begin{enumerate}
	\item The best possible resolution from the measurements
	\item Is robust to even heavy noise in the measurements.
	\item Is able to scale to big data problems.
\end{enumerate}

There is no reconstruction algorithm that solves all three problems. 

We need distributed algorithms.

The CLEAN algorithm, staple in radio astronomy.
Is robust and so far one of the faster algorithms

Big data problem is not sovled. Not every part is of the image reconstruction is distributed


This Project focuses on the third point, distributed image reconstruction with real-world meerkat observations
Received from SARAO for the purpose of algorithmic validation.
