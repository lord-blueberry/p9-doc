\section{The image reconstruction problem of radio interferometers}
Astronomy observe the sky in the radio spectrum.
measure the sky with high angular resolution.
What is angular resolution. arc-minutes, arc-seconds.
Angular resolution for radio antennas depend on the antenna dish diameter. Become too expensive
Radio interferometers go around this problem. They use several antennas together to act like one large antenna. Achieve higher angular resolution than single-dish instruments with cheaper hardware.

But there are drawbacks. First, the interferometer does not measure the sky in pixels, but in Fourier components. Each antenna pair measures a single Fourier component of the sky. The observed image has to be reconstructed from the measurements. And second, radio interferometers produce a large volume of Fourier measurements. 
MeerKAT new radio interferometer in the desert of "", measures 41 million Fourier components at each time interval, over possibly several hours of observation.
Reconstructing an image from the Fourier measurements is a big data problem.

At first glance, we might believe that the image reconstruction is trivial: The interferometer measures Fourier components, and the inverse Fast Fourier Transform is well-known. However two properties of the measured Fourier components make the image reconstruction difficult: The measurements are both noisy and incomplete.

The amplitude and phase of each measured Fourier component contains noise.
Radio waves on their way to earth experience distortions from various sources. and from the equipment. Changing conditions can create heavy interference, and the Fourier measurements are heavily corrupted by noise. Image reconstruction has to de-noise the measured Fourier components.

The measured Fourier components are incomplete. Although radio interferometers produce a large volume of measured Fourier components, they cannot measure every relevant Fourier component. There are Fourier components with non-zero amplitudes that the interferometer has not seen. The reconstruction algorithm has to find the observed image even though important Fourier components are missing from the measurements.

A reconstruction algorithm that can:
\begin{enumerate}
	\item The best possible resolution from the measurements
	\item Is robust to even heavy noise in the measurements.
	\item Is able to scale to big data problems.
\end{enumerate}

There is no reconstruction algorithm that solves all three problems. 

We need distributed algorithms.

The CLEAN algorithm, staple in radio astronomy.
Is robust and so far one of the faster algorithms

Big data problem is not sovled. Not every part is of the image reconstruction is distributed


This Project focuses on the third point, 
