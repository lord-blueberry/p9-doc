\section{Handling the Data Volume}\label{volume}
The new data volume is a challenge to process for both algorithms and computing infrastructure. Push for parallel and distributed algorithms. For Radio Interferometer imaging, we require specialized algorithms. The two distinct operations, non-uniform FFT and Deconvolution, were difficult algorithms for parallel or distributed computing.

The non-uniform FFT was historically what dominated the runtime \cite{}. Performing an efficient non-uniform FFT for Radio Interferometers is an active field of research\cite{offringa2014wsclean, pratley2018fast}, continually reducing the runtime costs of the operation. Recently, Veeneboer et al\cite{veenboer2017image} developed a non-uniform FFT which can be fully executed on the GPU. It speeds up the most expensive operation.

In Radio Astronomy, CLEAN is the go-to deconvolution algorithm. It is light-weight and compared to the non-uniform FFT, a cheap algorithm. It is also highly iterative, which makes it difficult for effective parallel or distributed implementations. However, compressed sensing based deconvolution algorithms can be developed with distribution in mind.

\subsection{Fully distributed imaging algorithm}
Current imaging algorithms push towards parallel computing with GPU acceleration. But with Veeneboer et al's non-uniform FFT and a compressed sensing based deconvolution, we can go a step further and create a distributed imaging algorithm. 




