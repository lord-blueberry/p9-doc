\section{Gradient/PSF approximation for distributed deconvolution} \label{gradients}

Distributed deconvolution, we would like to deconvovle patches of the final image independently from each other. 
Concept of seperability. Number of non-zero components.

The size of the $PSF$ tells us how far the patches have to be apart from each other to be independent.
Limiting factor for distribution is the size of the $PSF$. We would like to deconvolve pixels, or patches of the image independently from each other. But the $PSF$ does not let us do that. As soon as the $PSF$ overlap, they are not independent anymore.
Generally the $PSF$ spans over the whole image. 

But the $PSF$ for MeerKAT is peculiar.
In the center, the $PSF$ approximates a Gaussian function, which spans only over a fraction of the image. Indeed, with increasing number of visibility measurements, the $PSF$ becomes more and more like a Gaussian function.
Also, the values of the $PSF$ become smaller the further we walk away from the center. Although the $PSF$ spans over the whole image, its values become insiginificant.

MeerKAT has wide field of view observations with an increasingly Gaussian-like $PSF$. Can we use the significant fraction of the $PSF$ instead? Is this enough?
We have two operations, calculating the gradients of the image, and updating the gradients.
View of stochastic gradient descent. So if we calculate the gradient of a pixel with $-2 \langle residuals, PSF \rangle$, then we should be able to approximate the gradient with only a fraction fo the $SF$. But the pre-calculation of gradients is actually cheap, the thing. But what about updating the gradients?
We


Also,
Plus the major cycle framework: It already corrects for the fact that the $PSF$ we use is only an approximation of the true $PSF$. $w$-term changes the $PSF$ over the image. 



The answer to this is yes, we can. In Section \ref{results:gradients} we demonstrate the speedup of the deconvolution with only a fraction of the $PSF$ on a real world MeerKAT dataset. In this Section, we show how this can be done. 



The problem is arriving at the same optimum. Not trivial.



\subsection{Difficulty in approximating the $PSF$}

In our coordinate descent deconvolution, we use the $PSF$ in two steps. Before we run coordinate descent, we pre-calculate the gradient for each pixel with a correlation operation. During coordinate descent deconvolutions, we update the gradient-map directly. Two opposing solutions: We use the full $PSF$ to pre-calculate the gradients, but only update the gradients with a fraction of the $PSF$. That way we always start from the correct gradients, but get less accurate over coordinate descent deconvolutions.

\subsection{Approximate gradient update}

So we use the full $PSF$

only update with a fraction of the $PSF$

$PSF$ squared update. Scaling.
So we become less and less accurate


\subsection{Approximate deconvolution}

We never use the full PSF

But the problem of gradient magnitude. 

Change lambda. We do an approximate deconvolution. 


\subsection{Major Cycle convergence}
\cite{clark1980efficient} and the question on how many iterations per major cycle
Putting it all together

We have the Minor Cycle, which is easy to converge.

Coordinate Descent Path optimization \cite{friedman2010regularization}
Danger that CD takes too many pixel into a Major Cycle. Lower bound per iteration, PSF sidelobe
can still be too low, danger when many psf sidelobes overlap