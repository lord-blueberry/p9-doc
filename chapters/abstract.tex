\section*{Abstract}
The images, which a radio interferometer observes, has to be reconstructed by an algorithm. In the reconstruction process, an algorithm has to account for various artifacts in the measurement data of a radio interferometer. New radio interferometers create increasingly large data volumes, which put an ever increasing computational load on the reconstruction algorithm. For new radio interferometers, like the MeerKAT instrument in South Africa, the reconstruction algorithm has to be distributed to multiple machines. However, state-of-the-art algorithms like multi-scale CLEAN were developed without distributed computing in mind.

In this work, we move towards distributed image reconstruction for radio interferometers. We developed a novel reconstruction algorithm based on parallel coordinate descent, which can be easily extended to a distributed setting. Parallel coordinate descent alone is significantly slower than state-of-the-art algorithms. We developed a new approximation method inspired by Clark CLEAN, which speeds up the run time of our parallel coordinate descent algorithm. We compare the run time and reconstruction quality of our parallel coordinate descent algorithm with multi-scale CLEAN on a real-world MeerKAT observation. On our test the total run time was a factor of 3 faster than multi-scale CLEAN with a comparable reconstruction quality.

Our parallel coordinate descent algorithm is in the proof-of-concept stage. It does not account for all known artifacts found in the measurement data of modern radio interferometers. Furthermore we compared our algorithm on a single real-world observation. Future work would have to extend our parallel coordinate descent algorithm, and compared to state-of-the-art algorithms on different observations.



