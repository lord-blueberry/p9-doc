\section{Conclusion}
In this project, we developed our own image reconstruction pipeline in .Net Core. We implemented the Image Domain Gridder\cite{veenboer2017image} and developed two deconvolution algorithms: A serial and a parallel coordinate descent deconvolution algorithm in the Major/Minor cycle architecture. 

For this project, we developed the hypothesis that a deconvolution algorithm in the Major/Minor cycle architecture may be speed up by using an approximate $PSF$ for deconvolution. The deconvolution algorithm already uses an approximation of the true $PSF$. We showed that the $PSF$ can be further approximated, which simplifies the deconvolution problem for parallel and distributed computing. In this project, we created a novel $PSF$ approximation scheme for the Major/Minor cycle architecture, and implemented a parallel coordinate descent algorithm which can exploit the approximated  $PSF$ for a significant speedup. On our test, the parallel coordinate descent algorithm is estimated to outperform standard CLEAN in convergence speed.

The parallel coordinate descent algorithm can use modern non-blocking instructions, running in parallel deconvolutions with little communication overhead. Our implementation is optimized for the CPU on a shared-memory system. The algorithm can easily be extended for the distributed setting, where different machines deconvolve the image in parallel. The parallel coordinate descent algorithm showed competitive convergence speeds in comparison to CLEAN on the CPU. These results may be further improved by using GPU acceleration for the parallel algorithm.

The degree of parallelism scales with the problem size for the parallel coordinate descent algorithm. Large deconvolution problems can be efficiently solved by adding additional processors. This algorithm has the potential to scale with the planned expansion of MeerKAT to SKA-Mid. However, the imaging problem will also change together with MeerKAT's expansion. It remains to be seen if the parallel coordinate descent algorithm is competitive on the SKA-Mid imaging task.

We tested our $PSF$ approximation scheme and parallel coordinate descent algorithm on a real-world observation of MeerKAT. The approximation scheme and parallel algorithm are not inherently limited to a single instrument. However, our $PSF$ approximation scheme tends to be more effective on medium- to high-frequency observations. The $PSF$s of low-frequency observations cannot be approximated as efficiently. In turn, the parallel coordinate descent algorithm may not achieve the same speedups on low-frequency observations.

%The $PSF$ approximation scheme does not speed up any algorithm. For the serial coordinate descent algorithm, the $PSF$ approximation did result in speedup, but not enough to compete with CLEAN on convergence speed. One needs specialized reconstruction algorithms to exploit the $PSF$ approximation we developed in this project. The parallel coordinate descent algorithm can exploit the approximate $PSF$ for a significant speedup.

We used the elastic net regularization. To our knowledge, it is not widely used in the radio astronomy community. On our test, the elastic net regularization created more plausible reconstructions of radio sources than multi-scale CLEAN. The elastic net was also more sensitive to calibration errors in the image. This is a similar behavior to over-complete regularizations, which are often used for radio interferometric image reconstructions. However, elastic net is a significantly simpler regularization. Leading to a simpler reconstruction algorithm in a parallel or distributed setting. We demonstrated with the parallel coordinate descent algorithm that an elastic net regularized image can be reconstructed efficiently. These results suggests the elastic net regularization may be a viable alternative for radio interferometric image reconstruction.



T

  Reconstruction quality, comparable to multi-scale CLEAN. But also leading to more plausible structures in the model image. 
Potentially super-resolution.
But more problems when the image contains calibration errors.
Elastic net has a similar behavior to other over-complete representations. But the elastic net regularization is significantly simpler, only affecting a single pixel independently of its neighbors. This simplifies parallelization and distribution of the reconstruction. 
The parallel coordinate descent algorithm we developed can efficiently reconstruct an elastic net regularized image.

How useful our parallel coordinate descent algorithm is kinda depends on the usefulness of elastic net. Its with the elastic net regularization that the parallel algorithm is as fast as standard CLEAN. In this project, we tested the elastic net regularization only on a single real-world observation.
Elastic net has to be compared on more observations to make sure it is a useful regularization.


Results on narrow-band imaging only. Speedup is significant, but the more difficult question is how it compares in the wide-band imaging case.


 
